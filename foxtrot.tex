\documentclass[a4paper,UKenglish]{lipics-v2016}
%This is a template for producing LIPIcs articles. 
%See lipics-manual.pdf for further information.
%for A4 paper format use option "a4paper", for US-letter use option "letterpaper"
%for british hyphenation rules use option "UKenglish", for american hyphenation rules use option "USenglish"
% for section-numbered lemmas etc., use "numberwithinsect"

\usepackage{microtype}%if unwanted, comment out or use option "draft"

%\graphicspath{{./graphics/}}%helpful if your graphic files are in another directory

\bibliographystyle{plainurl}% the recommended bibstyle

% Author macros::begin %%%%%%%%%%%%%%%%%%%%%%%%%%%%%%%%%%%%%%%%%%%%%%%%
\title{First-Passage Percolation Time on Hypercubes}

\titlerunning{First-Passage Percolation Time}

\author{Luc Devroye, Abbas Mehrabian, and Pat Morin}
%\author[1]{Pat Morin}
%\affil[1]{Carleton University School of Computer Science, Ottawa, Canada \\
%   \texttt{morin@scs.carleton.ca}}
%\authorrunning{P. Morin} %mandatory. First: Use abbreviated first/middle names. Second (only in severe cases): Use first author plus 'et. al.'

\Copyright{L. Devroye, A. Mehrabian, and P. Morin}%mandatory, please use full first names. LIPIcs license is "CC-BY";  http://creativecommons.org/licenses/by/3.0/

%\subjclass{Dummy classification -- please refer to \url{http://www.acm.org/about/class/ccs98-html}}% mandatory: Please choose ACM 1998 classifications from http://www.acm.org/about/class/ccs98-html . E.g., cite as "F.1.1 Models of Computation". 
%\keywords{Dummy keyword -- please provide 1--5 keywords}% mandatory: Please provide 1-5 keywords
%% Author macros::end %%%%%%%%%%%%%%%%%%%%%%%%%%%%%%%%%%%%%%%%%%%%%%%%%
%
%Editor-only macros:: begin (do not touch as author)%%%%%%%%%%%%%%%%%%%%%%%%%%%%%%%%%%
%\EventEditors{John Q. Open and Joan R. Acces}
%\EventNoEds{2}
%\EventLongTitle{42nd Conference on Very Important Topics (CVIT 2016)}
%\EventShortTitle{CVIT 2016}
%\EventAcronym{CVIT}
%\EventYear{2016}
%\EventDate{December 24--27, 2016}
%\EventLocation{Little Whinging, United Kingdom}
%\EventLogo{}
%\SeriesVolume{42}
%\ArticleNo{23}
% Editor-only macros::end %%%%%%%%%%%%%%%%%%%%%%%%%%%%%%%%%%%%%%%%%%%%%%%


\DeclareMathOperator{\E}{E}
\DeclareMathOperator{\exponential}{exponential}
\DeclareMathOperator{\binomial}{binomial}

\newcommand{\zero}{\mathbf{0}}
\newcommand{\one}{\mathbf{1}}
\newcommand{\etal}{\emph{et al}}

\begin{document}

\maketitle

\begin{abstract}
We give a simple proof of the following statement: If one puts independent
exponential mean 1 edge weights on the edges of a $d$-cube, then the
expected weight of the lightest path from $(0,\ldots,0)$ to $(1,\ldots,1)$
is $O(1)$. 
\end{abstract}

\section{Introduction}

The $d$-cube $Q_d$ is the graph with vertex set $V(Q_d)=\{0,1\}^d$
and whose edge set $E(Q_d)$ contains the edge $uv$ if and only if the
Hamming distance between $u$ and $v$ is exactly 1.  Assign an independent
$\exponential(1)$ \emph{edge weight} to each edge of $Q_d$.  These weights
define a lightest path metric between the vertices of $Q_d$, where
$w(u,v)$ denotes the weight of the lightest path from vertex $u$ to vertex
$w$.  What is the expected weight $\E[w(\mathbf{0},\mathbf{1})]$ of the
lightest path from $\mathbf{0}=(0,\ldots,0)$ to $\mathbf{1}=(1,\ldots,1)$?

In this note, we offer a simple proof that $\E[w(\mathbf{0},\mathbf{1})]
\in O(1)$;  although the number of edges in any path from $\mathbf{0}$
to $\mathbf{1}$ is at least $d$, the expected weight of the lightest path
does not increase with $d$.  This result is not new.  Indeed, this type of
question is central in the study of \emph{first-passage percolation} as
introduced by Hammersley and Welsh in 1965 \cite{hammersley.welsh:first}
and recently surveyed by Auffinger \etal\ \cite{auffinger.damron.ea:50}.

The question we consider here was first asked by Aldous
\cite[Section~G7]{aldous:probability} and first answered by Fill and
Pemantle \cite{fill.pemantle:percolation} who showed that the
weight of the lightest monotone path from $\zero$ to $\one$ converges
in probability to $1$ as $d\rightarrow\infty$.  The most recent result
on this problem is due to Martinsson \cite{martinsson:unoriented},
who shows that $w(\mathbf{0},\mathbf{1})$ converges in probability to
$\ln(1+\sqrt{2})\approx 0.881$ as $d\rightarrow\infty$.  The difference
between these two results is that Fill and Pemantle's paths are \emph{monotone}---they have exactly $d$ edges---while Martinsson's paths may have more than
$d$ edges.

The proof we present here is (arguably) simpler and more accessible
to a computer science audience than either of the proofs discussed
above.  On the other hand, our proof only gives an $O(1)$ upper bound
(approximately $303.61$) on the weight of the lightest path from $\zero$
to $\one$ and doesn't give any lower bound.  It also does not guarantee
a monotone path; it produces paths using roughly $3d/2$ edges.

One nice feature of this new proof is that it employs a natural greedy
strategy that results in an algorithm for finding an $O(1)$ weight
path that runs in $O(d^4)$ time.  In distributed computing terms, this
algorithm is 3-local, it can be implemented by an agent that only has
information about edge weights in a neighbourhood of radius 3 about the
current vertex.

\section{Review of Probability Concepts}

Recall that an $\exponential(\lambda)$ random variable $X$ has a
distribution defined by
\[
   \Pr\{X > x\} = e^{-\lambda x} \enspace,\enspace x \ge 0 \enspace ,
\]
and has expected value
\[
   \E[X] = \int_0^\infty \Pr\{X>x\}\,dx = \int_0^\infty e^{-\lambda x}\,dx = 1/\lambda
\]
If $X_1,\ldots,X_k$ are independent $\exponential(1)$ random variables,
then their minimum is an $\exponential(k)$ random variable.
\[
   \Pr\{\min\{X_1,\ldots X_k\} > x\} = \Pr\{\text{$X_1>x$, $X_2>x$,\ldots, and $X_k>x$}\} = \left(e^{-x}\right)^k = e^{-kx} \enspace .
\]
At one point, we will make use of a simple Chernoff bound for binomial
random variables. If $B$ is a binomial random variable with expected
value $\E[B]$, then
\begin{equation}\label{eq:chernoff}
   \Pr\{B < \E[B]/2\} \le e^{-\E[B]/8} \enspace .
\end{equation}

\section{Some Intuition}
\label{sec:intuition}

Before continuing, we first describe a na\"{\i}ve greedy algorithm that
does not quite work.  Suppose that, to route from $\zero$ to $\one$ we
employ the strategy of repeatedly taking the lightest edge that
takes us closer to $\one$.  At the zeroth step, there are $d$ edges to
choose from, so the lightest one will have a weight that is the minimum of
$d$ $\exponential(1)$ random variables, i.e., it is an $\exponential(d)$
random variable and its expected value is $1/d$.  At the first step,
there are $d-1$ edges to choose from, so the expected weight of the edge
we choose is $1/(d-1)$. In general, the expected weight of an edge we
choose at the $i$th step is $1/(d-i)$.  Thus, the expected weight of
the edges crossed by this greedy algorithm is
\[
   \sum_{i=0}^{d-1} 1/(d-i) = \sum_{i=1}^d 1/i \le \ln d + 1 \enspace . 
\]
This is not quite the $O(1)$ bound we are hoping for, but it is significantly better than the obvious $O(d)$ bound.

The problem with this greedy algorithm is that it works well for the
first $d/2$ steps, but the cost of each step increases as it gets closer
to $\one$, eventually yielding the $d$th harmonic number.  Our solution
to this problem is to employ a \emph{foxtrot} in the second $d/2$ steps,
in which we repeatedly take a step away from $\one$ followed by two steps
toward $\one$.  In the $i$th stage, this allows us to choose from among
$i(d-i)^2$ different paths of length three instead of being restricted
$d-i$ paths of length 1. Next, we prove a lemma that allows us to analyze these foxtrot steps.


\section{Trees of Height 3}
\label{sec:abc}

The following result, depicted in Figure~\ref{fig:tree}, studies a
first-passage percolation problem on a tree of height three.

\begin{figure}
   \begin{center}
     \begin{tabular}{cc}
       \includegraphics{figs/tree-1} & \includegraphics{figs/tree-2} \\
       (a) & (b)
     \end{tabular}
   \end{center}
   \caption{(a)~The tree $T$ for $a,b,c=4,3,2$.  After removing edges of weight greater than $t/3$, we study the component $T'$ containing the root of $T$.}
   \label{fig:tree}
\end{figure}


\begin{lemma}\label{lem:abc}
Let $a,b,c\ge 1$ be integers and let $T$ be a rooted tree of height
three of whose root has $a$ children, each of which has $b$ children,
each of which has $c$ children.  Assign an $\exponential(1)$ edge weight
to each edge of $T$ and let $\rho(T)$ denote the weight of the lightest
root-to-leaf path in $T$.  Then
\[
    \Pr\{\rho(T) > t\} \le e^{-at/64} + e^{-bat^2/1024} + e^{-cbat^3/768} \enspace .
\] 
\end{lemma}

Our only use for Lemma~\ref{lem:abc} is to upper bound the expected
value of $\rho(T)$.  We do this now, before proving
Lemma~\ref{lem:abc}.

\begin{corollary}\label{cor:expectation}
  Let $a$, $b$, $c$, $T$, and $\rho(T)$ be defined as in Lemma~\ref{lem:abc},
  with $a\ge b\ge c\ge 1$.  Then $\E[\rho(T)] \le C/(abc)^{1/3}$, for
  $C=64 + 16\sqrt{\pi} + 16(1/12)^{2/3}\Gamma(1/3)$.
\end{corollary}

\begin{proof}
   Recall that, for any non-negative random variable $X$, $\E[X]=\int_0^{\infty} \Pr\{X>x\}\,dx$.  Therefore,
  \begin{align*}
    \E[\rho(T)] & = \int_0^{\infty} \Pr\{\rho(T)>t\}\,dt \\
                & \le \int_0^{\infty}\left( e^{-at/64} + e^{-bat^2/1024} + e^{-cbat^3/768}\right)\,dt \\
                & = \frac{64}{a} + \frac{16\sqrt{\pi}}{\sqrt{ab}} + \frac{16(1/12)^{2/3}\Gamma(1/3)}{\sqrt[3]{abc}} \\
                & \le \frac{64 + 16\sqrt{\pi} + 16(1/12)^{2/3}\Gamma(1/3)}{\sqrt[3]{abc}}
  \end{align*}
  where the last inequality uses the assumption that $a\ge b\ge c$.
\end{proof}


\begin{proof}[Proof of Lemma~\ref{lem:abc}]
  For large values of $t$, the proof is simple. In particular, if $t\ge 6\ln 3$,
  then we observe that $T$ contains $a$ edge-disjoint root-to-leaf paths.
  For one of these paths to have weight greater than $t$, at least one of
  its three edges must have weight greater than $t/3$.  The probability
  that this occurs (for a single path) is at most $3e^{-t/3}$. Since
  the paths are edge-disjoint, their weights are independent, so the
  probability that it occurs for all $a$ paths is therefore at most
  \begin{align*}
     (3e^{-t/3})^a = (e^{\ln 3-t/3})^a = (e^{(\ln 3/t-1/3)t})^a \le (e^{-t/6})^a = e^{-at/6}\enspace ,
  \end{align*}
  where the inequality uses the assumption that $t\ge 6\ln 3$.

  We now move on to the interesting case, where
  $0\le t < 6\ln 3$. 
  Imagine removing every edge of $T$ having weight greater than $t/3$
  to obtain a forest $F$ and let $T'$ be the tree in $F$ that contains
  the root of $T$.  For each $i\in\{0,1,\ldots,3\}$, let $N_i$ denote
  the number of nodes of $T'$ having depth $i$.
  Observe that, if $N_3\ge 1$, then there is a root-to-leaf path in $T$
  of weight at most $t$.  Therefore the rest of the proof is devoted to
  upper bounding $\Pr\{N_3=0\}$.

  Observe that $N_1$ is a $\binomial(a,1-e^{-t/3})$ random variable.
  The probability $1-e^{-t/3}$ is a bit unwieldy so we observe that,
  in the range $0\le t < 6\ln 3$, $1-e^{-t/3}\ge t/8$.  Therefore, we
  can lower bound the expected value
  \[
     \E[N_1] = a(1-e^{-t/3}) \ge at/8 \enspace .
  \]
  Since $N_1$ is binomial, 
  by \eqref{eq:chernoff},
  \[
     \Pr\{ N_1 < at/16 \} \le e^{-at/64} \enspace .
  \]
  Now, conditioned on $N_1$, $N_2$ is a $\binomial(bN_1, 1-e^{-t/3})$ random variable and
  \[
      \E[N_2\mid N_1\ge at/16] \ge bat(1-e^{-t/3})/16 \ge bat^2/128 \enspace .
  \]
  Again, \eqref{eq:chernoff} yields
  \[
      \Pr\{N_2 < bat^2/256\mid N_1 \ge at/16\} \le e^{-bat^2/1024} \enspace .
  \]
  Now, conditioned on $N_2$, $N_3$ is a $\binomial(cN_2,1-e^{-t/3})$
  random variable but for this last step we don't need Chernoff's help:
  \[
      \Pr\{N_3 = 0 \mid N_2\ge bat^2/256\} \le (e^{-t/3})^{cbat^2/256}
          = e^{-cbat^3/768} \enspace .
  \]
  Summarizing,
  \begin{align*}
     \Pr\{\rho(T) > t\} 
                    & \le \Pr\{N_3 = 0\} \\
                    & \le \Pr\{N_3 = 0\mid N_2 \ge bat^2/256\} \\
                    &\quad{}+ \Pr\{N_2 < bat^2/256 \mid N_1 \ge at/16\} \\
                    &\quad{}+ \Pr\{N_1 < at/16\} \\
        & \le e^{-at/64} + e^{-bat^2/1024} + e^{-bat^3/768} \enspace .  \qedhere
  \end{align*}
\end{proof}


\begin{remark}
We note that the result of Lemma~\ref{lem:abc} also holds in
a slightly more general setting, an example of which is illustrated in
Figure~\ref{fig:dag}. In particular, we can identify groups
of leaves of $T$ arbitrarily to obtain a directed acyclic graph $D$ in which the
leaves of $T$ become sinks in $D$. The lemma then gives bounds on the
probability that the weight of the lightest root-to-sink path exceeds $t$.
\end{remark}

\begin{figure}
   \begin{center}
     \begin{tabular}{cc}
       \includegraphics{figs/dag-1} & \includegraphics{figs/dag-2} \\
       (a) & (b)
     \end{tabular}
   \end{center}
  \caption{The result of Lemma~\ref{lem:abc} holds in a slightly more general setting, in which some of the leaves of $T$ are identified.}
  \label{fig:dag}
\end{figure}

\section{The Proof}


\begin{theorem}
  Let $Q_d$ be the $d$-cube equipped with independent $\exponential(1)$
  edge weights.  Then the expected weight of the lightest path from $\zero=(0,\ldots,0)$
  to $\one=(1,\ldots,1)$ is $O(1)$.
\end{theorem}

\begin{proof}
  For each $i\in\{0,\ldots,d\}$, let $L_i$ denote set of vertices of $Q_d$
  whose distance from $\zero$ is exactly $i$, so that $\zero\in L_0$ and
  $\one\in L_d$.  We use a greedy strategy to find a path from $\zero$
  to $\one$.  To get from a vertex $u\in L_i$ to some vertex in $L_{i+1}$
  the strategy does one of the following two things:
  \begin{enumerate}
     \item If $i < d/2$, then the algorithm traverses the lightest edge
     joining $u$ to some vertex in $L_{i+1}$.  The weight of this edge is
     the minimum of $d-i$ independent $\exponential(1)$ random variables,
     so the expected weight of this edge is $1/(d-i)\le 2/d$.

     \item If $i \ge d/2$, we consider the $i(d-i)^2$ paths $uxyz$ with
     $x\in L_{i-1}$, $y\in L_i\setminus\{u\}$, and $z\in L_{i+1}$ and
     traverse the lightest such path.  See Figure~\ref{fig:foxtrot}. This
     set of paths has the structure of the DAG described in the
     remark at the end of Section~\ref{sec:abc}:  The root has $i$
     outgoing edges and the nodes at depth 1 and 2 each have $d-i$
     outgoing edges.  Therefore, by Corollary~\ref{cor:expectation},
     the expected weight of the three edges traversed in this step is
     at most $Ci^{-1/3}(d-i)^{-2/3}$.
  \end{enumerate}
  \begin{figure}
    \begin{center}
       \includegraphics{figs/foxtrot}
    \end{center}
    \caption{A foxtrot step considers all paths $uxyz$ with $x\in L_{i-1}$, $y\in L_i\setminus\{u\}$ and $z\in L_{i+1}$.}
    \label{fig:foxtrot}
  \end{figure}
  Therefore, the expected weight of the entire path found by this
  algorithm is at most
  \begin{align*}
    \mu & \le \sum_{i=0}^{\lfloor (d-1)/2\rfloor} 2/d 
               + \sum_{i=\lceil d/2\rceil}^{d-1} Ci^{-1/3}(d-i)^{-2/3} \\
        & \le 2 
               + C\left(\sum_{i=\lceil d/2\rceil}^{d-1} i^{-1/3}(d-i)^{-2/3}\right) \\
        & \le 2 
               + C(d/2)^{-1/3}\left(\sum_{i=\lceil d/2\rceil}^{d-1} (d-i)^{-2/3}\right) \\
        & = 2 
               + C(d/2)^{-1/3}\left(\sum_{k=1}^{\lfloor d/2\rfloor}k^{-2/3}\right) \\
        & \le 2
               + C(d/2)^{-1/3}\left(1+\int_1^{d/2}x^{-2/3}\,dx\right) \\
        & = 2
               + C(d/2)^{-1/3}\left(3(d/2)^{1/3}-3\right) \\
        & \le 2
               + 3C \enspace . \qedhere
  \end{align*}
\end{proof}

\section{Discussion}

Our proof works for any non-negative continuous edge weight distribution
that has density 1 around zero and whose tail decays exponentially.
The first property ensures the minimum of $k$ independent random
samples from the distribution has expected value $O(1/k)$. The second
property ensures that there is are constants $T,c>0$ such that, for all
$t>T$, $\Pr\{X > t\}\le e^{-ct}$.  Besides the $\exponential(\lambda)$
distribution for constant $\lambda$, another notable example is the
uniform distribution over the interval $[0,1]$.

The weight of the path found in our proof is the sum of $d$ random
variables. The first $d/2$ of these are independent $\exponential(1)$. The
second $d/2$ can be split into two subsets (the odd steps and the even
steps) that are each independent.  Using standard methods for deriving
concentration inqualities along with the fact that Lemma~\ref{lem:abc}
gives an exponential tail bound on $\rho(T)$, it is possible to show
that, for any $\delta >0$, $\Pr\{\mu\ge (1+\delta)(1+3C)\}\rightarrow 0$
as $d\rightarrow\infty$.  Unfortunately, the rate of this convergence
is not quite enough to prove that with high probability there is a path
of weight $O(1)$ from $\zero$ to \emph{every} vertex of $Q_d$.
This latter fact is something that Fill and Pemantle's proof does manage
to show \cite{fill.pemantle:percolation}.

A conjecture of Aldous \cite[Conjecture~G7.1]{aldous:probability}
was the original motivation for the work of Fill and Pemantle.  In his
discussion of this conjecture, Aldous describes the na\"{\i}ve greedy
algorithm from Section~\ref{sec:intuition} and shows that it produces a
path whose expected weight is the $d$th harmonic number.  In their review
of previous work, Fill and Pemantle \cite{fill.pemantle:percolation}
point out that similar results on percolation were proven for complete
binary trees \cite{pemantle:phase}.  24 years later, our proof shows
that the result for hypercubes is a consequence of the na\"{\i}ve greedy
algorithm and a result for trees of height three.

The proof of Fill and Pemantle \cite{fill.pemantle:percolation}
and an unpublished proof of Bollob\'as \etal\
\cite{balister.bollobas.ea:first-passage} both work by a careful analysis
of the $d!$ monotone paths from $\zero$ to $\one$, and the ways in which
pairs of these paths can overlap.  Fill and Pemantle require this because
they use (a variant of) the second moment method, while Bollob\'as \etal\
use a lemma due to Janson that also has conditions on the interactions
between pairs of objects. Our proof sidesteps all of this.

Martinsson's proof \cite{martinsson:unoriented} works by relating this
problem to a so-called \emph{branching translation process} and then
using a variety of advanced probabilistic tools to study this process.
This branching translation process was used by Fill and Pemantle's
original work to provide a lower bound on $w(\zero,\one)$.

Finally, we point out that the first-passage percolation time in a
graph $G$ with i.i.d.\ exponential edge weights is closely related
to the the maximum number of edges, $h(G,s)$, in the lightest
path from some vertex $s$ any other vertex of $G$.  Devroye \etal\
\cite{devroye.dujmovic.ea:notes} describe a relationship between
first-passage percolation time, the number of simple paths in $G$ and
$h(G,s)$ and use this to derive bounds on $\E[h(G,s)]$ that are tight
for many classes of graphs.


\section*{Acknowledgements}

We would like to thank Alan~Frieze and G\'abor Lugosi for helpful
discussions and suggestions.  This work was born of discussions that
took place during the Workshop on Random Geometric Graphs and Their
Applications to Complex Networks, at the Banff International Research
Station, November 6--11, 2016.

\bibliography{foxtrot}

\end{document}



